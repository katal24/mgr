\chapter{Pairwise Comparisons method}
\label{sec:pcMethod}
  \section{PC matrix}
	\label{subsec:macierzPC}
	
	The \textit{PC method} is used to choose the best alternative from a set of concepts. However, this goal is achieved by comparing in pairs. A numerical value is assigned to each pair. It not only determines which alternative is preferred but also informs about the strength of this preference. In this way, the finite set of concepts $C=\left\{ c_{1},\ldots,c_{n}\right\} $ is transformed into a \textit{PC matrix} $M=\left(m_{ij}\right)$, where $m_{i,j}\in \mathbb{R}$ and $i,j\in\left\{ 1,\ldots,n\right\}$. The PC matrix for $n$ concepts is following:
$$
M = 
\left(
\begin{array}{cccc}
	1 & m_{12} & \dots & m_{1n}\\
	m_{21} & 1 & \dots & m_{2n}\\
	\vdots & \vdots & \ddots & \vdots\\
	m_{n1} & m_{n2} & \dots & 1\\ 	
\end{array}
\right)
$$

	It is worth noting that the values $m_{ij}$ and $m_{ji}$ represent the same pair. Therefore, one should expect that $m_{ji}=\frac{1}{m_{ij}}$. If
	\begin{equation} 
		\forall _{i,j\in\left\{ 1,\ldots,n\right\}} ~m_{ij}=\frac{1}{m_{ji}},
	\end{equation}
		the matrix is called a \textit{reciprocal}.
	
	Calculations on PC matrices are usually performed when the matrix contains all elements. For the matrix with dimension $n$ by $n$ there are $n(n-1)$ values. Such matrix is called \textit{complete}. If one or more values are missing, the matrix is called \textit{incomplete}.

		% Calculations on PC matrices are usually performed when the matrix contains all elements. For the matrix with dimension n by n there are n(n-1) values. Such matrix is called complete. If one or more values are missing, the matrix is called incomplete.

  \section{Weight vector}
	\label{subsec:wektorWag}
The PC matrix provides data allowing the method for calculating the ranking vector $\mju$. This vector defines the preferential values assigned by the function $\mu:C\rightarrow \mathbb{R}$ to each alternative $c_i$ in $C$. The vector $$\mu=\left[\mu\left(c_{1}\right), \ldots, \mu\left(c_{n}\right)\right]$$ formed in this way is called \textit{the weight vector} or \textit{the priority vector} (see Fig.2.1). By choosing the alternative with the highest preferential value in $\mu$ one can easily find which alternative win the ranking. 
\begin{center}
\begin{figure}[ht]
\begin{tabular}{cm{0.4\linewidth}c}
$ ~~~~~~\left(\begin{array}{cccc}
	1 & m_{12} & \dots & m_{1n}\\
	m_{21} & 1 & \dots & m_{2n}\\
	\vdots & \vdots & \ddots & \vdots\\
	m_{n1} & m_{n2} & \dots & ~~1\\ 	
\end{array}\right)$ & \includegraphics[scale=1.9]{fig3.png} & ~~~~$\left(
\begin{array}{c}
	\mu\left(c_1\right)\\
	\mu\left(c_2\right)\\	
	\vdots \\
	\mu\left(c_n\right)\\	
\end{array}
\right)$ \\
\end{tabular}
\caption{Diagram of calculating \textit{the weight vector} based on the article \cite{Kulakowski2014}}
\end{figure}
\end{center}

There are many ways to calculate the vector $\mu$. Among the popular ones are the method using the matrix's eigenvalues or the method based on geometric means \cite{SAATY1998}.

\section{Inconsistency}
\label{subsec:inconsistency}
The second important parameter describing the PC matrix is \textit{consistency}. The matrix is consistent if 
	\begin{equation}
		\label{eq:consistent}
		\forall _{i,j,k \in\left\{ 1,\ldots,n\right\}}~  m_{ik}=m_{ij}m_{jk}.
	\end{equation}
Three numbers that should meet this assumption are called a \textit{triad}.

If the matrix is consistent and the weight vector $\mu$ is computed, then for each of the variables $i,j$, where $1\leq i,j\leq n$ meets the equation: 
	\begin{equation} 
		\label{eq:consistent2}		
		m_{ij}=\frac{\mu_{i}}{\mu_{j}}.
 	\end{equation}
In practice, it is very rare for a matrix $M$ to be completely consistent. In the long history of the PC method, many methods were developed to calculate inconsistencies. Many of them are based directly on the definition of consistency~(\ref{eq:consistent}), some methods use the eigenvalues of the matrix, others are based on the assumption that each fully consistent matrix fulfills the condition~(\ref{eq:consistent}).