\chapter{Introduction}
\label{cha:wprowadzenie}

\section{Pairwise Comparisons method}
\label{sec:metodaPorowan}
People have made decisions for ages. Some of them are very simple and come easily but others, more complicated, require deeper analysis. It happens when there are many compared objects, which are complex and the selection criterion is hard to measure precisely. Fortunately, the development of mathematics brought an interesting tool - \textit{The Pairwise Comparisons (PC) Method}. The first case of using the method (in a very simple version) is the election system described by \textit{Ramond Llull} \cite{Colomer2013} in the thirteenth century. Its rules were based on the fact that the candidates were pairwise compared with each other and the winner was the one who won in the largest number of direct comparisons. The method was reinvented in the eighteenth century by \textit{Condorcet} and \textit{Bord} \cite{Kulakowski2016} as they proposed it in their voting system. In the twentieth century, the method found the application in the theory of social choice, the main representatives of which were the Nobel prize winners \textit{Keneth Arrow} \cite{Arrow} and \textit{Amartya Sen} \cite{Sen}. The current shape of the method was influenced by the changes introduced by \textit{Fechner} and then refined by {Thrustone} \cite{Thurstone1994}. However, the breakthrough was the introduction to the method \textit\textit{The Analytic Hierarchy Process (AHP)} by \textit{Saaty} \cite{Saaty2008}, which allowed to compare many more complex objects and create a hierarchical structure.The main aim of this paper is to check which method of calculating the inconsistency is the best in this case. I order to do it, a series of tests was carried out on various known inconsistency indexes, taking into account many different parameters: the matrix size, the amount of missing data and the level of inconsistency. The results of the research are included in this paper.

The PC method is based on the assumption that it is not worth comparing all objects at the same time. It is better to compare them in pairs and then gather the results together. Such pairwise comparisons are much more intuitive and natural for a human being. How can one be sure that these judgments are consistent? Or what to do if some comparisons are missing? In such a case, is it worth taking the \textit{PC method} at all?

The answer to the first question is the concept of inconsistency introduced into the method. This paper tries to answer the next two questions - meaning to examine whether available methods for determining inconsistencies give reliable results when a part of the comparisons is missing. 

\section{The aim of the work}
\label{sec:celePracy}
The main aim of this paper is to check which method of calculating the inconsistency is the best when some comparisons are missing. In order to do it, a series of tests was carried out on various known inconsistency indexes, taking into account many different parameters: the matrix size, the amount of missing data and the level of inconsistency.
The tests were implemented in R language which properly fit into numerical calculations.

It was not knowN from the beginning what will be the result of this work. It was considered that one or more existing inconsistency indexes turn out appropriate also for incomplete matrices or the tests show that the inconsistency can not be calculated by these indexes. 
The results of the research are included in this study.

\section{Contents of the work}
\label{sec:zawartoscPracy}
The work includes the theoretical part and the description of conducted experiments and their results. It consists of six chapters.\\
The Second section shows the Pairwise Comparisons method and the problem of inconsistency. Understanding the basics is necessary to go into the further chapters.\\
In the third section sixteen available methods for calculating the inconsistency for the PC matrices are presented. \\
The fourth chapter shows ideas which allow to perform tests. These are modifications of existing inconsistency indexes which enable to adjust them to incomplete matrices and the algorithm which tests the quality of the modified indexes. \\
The results of experiments are presented and discussed in the fifth section.\\
The last chapter contains summary, conclusions and ideas for future researches.