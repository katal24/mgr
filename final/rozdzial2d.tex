\chapter{Pozostałe metody}
\label{sec:pozostałe}
W ciągu wielu lat rozwijania metody porównań parami i badań prowadzonych w tej dziedzinie, powstało wiele funkcji, które nie są bezpośrednio częścią PC, przyczyniają się jednak do weryfikacji prawidłowości działania metody, upraszczają pracę z macierzami, również tymi niekompletnymi, czy pozwalają sprawdzić otrzymane wyniki. W tym rozdziale przedstawię kilka z takich metod.
	
\section{Łączenie rankingów}
\label{subsec:aij}
Kiedy chcemy rozwiązać jakiś określony problem wykorzystując metodę porównań parami, możemy sami dokonać porównań lub poprosić o nie ekspertów w danej dziedzinie. To daje nam nadzieję, że wyniki będą obiektywne i odzwierciedlające rzeczywistość. Po otrzymaniu od nich macierzy PC, chcemy połączyć te tabele i utworzyć z nich jedną, która będzie odzwierciedlać preferencje wszystkich ekspertów. Drugą alternatywą jest obliczenie wektora wag dla każdej z otrzymanych macierzy, a następnie połączenie ich w jeden, sumaryczny wektor.

Aby tego dokonać możemy posłużyć się metodami \textit{Aggregating individual judgments and priorities (AIJ i AIP)}, które wprowadzili Forman i Peniwati~\cite{A13}. Proponują oni użycie funkcji, które wyliczają średnie arytmetyczne i geometryczne, zarówno dla macierzy, jak i dla wektorów. 

\begin{example}
$$M_{1} = 
\left(
\begin{array}{111}
	1 & 2 & 4 \\
	\frac{1}{2} & 1 & 2 \\
	\frac{1}{4} & \frac{1}{2} & 1 	
\end{array}
\right) \quad \quad \quad M_{2} = 
\left(
\begin{array}{111}
	1 & \frac{1}{3} & 4 \\
	3 & 1 & 9 \\
	\frac{1}{4} & \frac{1}{9} & 1 	
\end{array}
\right)  $$
Obliczamy macierz PC, która odzwierciedla obie macierze:
\begin{itemize}
\item z wykorzystaniem średniej arytmetycznej:
$$M_{art} = 
\left(
\begin{array}{111}
	1 & \frac{7}{6} & 4 \\
	1\frac{3}{4} & 1 & 5\frac{1}{2} \\
	\frac{1}{4} & \frac{11}{36} & 1 	
\end{array}
\right) 
$$
\item z wykorzystaniem średniej geometrycznej:
$$M_{geom} = 
\left(
\begin{array}{111}
	1 & 0.8164966 & 4 \\
	1.224745 & 1 & 4.242641 \\
	0.25 & 0.2357023 & 1 	
\end{array}
\right) $$
\end{itemize}
\end{example}

\section{Wektor wag a macierz PC}
\label{subsec:cop}
Przyjrzyjmy się jeszcze raz przedstawionemu już wcześniej warunkowi spójności macierzy: $m_{ik}~=~m_{ij}m_{jk}$. Wyprowadzenie tego wzoru jest proste i opiera się na spostrzeżeniu, że stosunek dwóch elementów wektora wag powinien być równy odpowiadającej im wartości w macierzy PC. Przykładowo, jeśli alternatywa druga ma wartość $0.4$, a alternatywa trzecia $0.2$, to spodziewamy się, że wartość porównania drugiej i trzeciej alternatywy wynosi $\frac{0.4}{0.2} = 2$. Szczegółową wersję spójności macierzy można więc przedstawić następująco:
	\begin{equation} \label{eq:2}
	a_{ij}a_{jk} = \frac{w_{i}}{w_{j}}\frac{w_{j}}{w_{k}} = \frac{w_{i}}{w_{k}} = a_{ik}.
\end{equation}	 	

Właśnie na tym spostrzeżeniu bazują Bana e Costa i Vansnick~\cite{A14} w swoich badaniach, wprowadzając dwa sposoby sprawdzania spójności i wykrywania miejsc, w których macierz nie jest spójna: \textit{(first and second Condition of Order Preservation - COP)}. Pierwszy sposób (COP1) sprawdza, czy wybór lepszej alternatywy w każdej parze przekłada się na większą wartość danej alternatywy w wektorze wag. Drugi sposób (COP2) jest bardziej dokładny i wprost weryfikuje warunek \ref{eq:2} dla każdej pary alternatyw.

\section{Odległość między wektorami}
\label{subsec:kendall}
Ciekawym algorytmem jest obliczanie odległości między wektorami. Sposób ten przedstawił Maurice Kendall~\cite{A15}. Opiera się on na idei sortowania bąbelkowego. W czasie sortowania tego rodzaju, kolejno porównujemy sąsiadujące ze sobą elementy i jeśli nie są we właściwej kolejności, to zamieniamy je miejscami. Czynność powtarzamy aż do momentu, gdy cała lista elementów jest posortowana. Kendall zaproponował algorytm zliczający ilość \textit{zamian miejscami}, które należałoby wykonać w danym wektorze, by stał się identyczny do drugiego wektora.

\section{Inne funkcje usprawniające pracę z macierzami PC}
\label{subsec:inne}
W bibliotece PairwiseComparisons znalazły się także inne funkcje, które bezpośrednio nie dotyczą metody porównań parami, ułatwiają jednak prace z macierzami, w szczególności z macierzami PC. Przykładem może być funkcja \textit{recreatePCMatrix}, która przyjmuje macierz z uzupełnionymi wartościami tylko powyżej przekątnej, a pozostałe elementy uzupełnia automatycznie.

\newpage
\section{Biblioteka PairwiseComparisons - pozostałe metody}
\label{subsec:pozostałe}

\textbf{Funkcje z biblioteki PairwiseComparisons pomocne w powyższych obliczeniach:}
\\

\begin{spacing}{1.1}
\noindent{\Large Agregacja macierzy/wektorów (średnia arytmetyczna) } \\ \begin{spacing}{0.5}  \end{spacing}

\textbf{Nazwa}\\  \emph{AIJadd} \\ \begin{spacing}{0.3}  \end{spacing}
 
\textbf{Opis}\\ Oblicza macierz lub wektor, którego elementy są średnią arytmetyczną przekazanych macierzy lub wektorów.\\  \begin{spacing}{0.3}  \end{spacing}
 
\textbf{Argumenty} \\
... -- lista macierzy lub wektorów tych samych wymiarów\\ \begin{spacing}{0.3}  \end{spacing}

\textbf{Zwracana wartość}\\ Średnia macierz/wektor \\ \begin{spacing}{0.3}  \end{spacing}\\

\\~\\
\begin{spacing}{1.1}
\noindent{\Large Agregacja macierzy/wektorów (średnia geometryczna) } \\ \begin{spacing}{0.5}  \end{spacing}

\textbf{Nazwa}\\  \emph{AIJgeom} \\ \begin{spacing}{0.3}  \end{spacing}
 
\textbf{Opis}\\ Oblicza macierz lub wektor, którego elementy są średnią geometryczną przekazanych macierzy lub wektorów.\\  \begin{spacing}{0.3}  \end{spacing}
 
\textbf{Argumenty} \\
... -- lista macierzy lub wektorów tych samych wymiarów\\ \begin{spacing}{0.3}  \end{spacing}

\textbf{Zwracana wartość}\\ Średnia macierz/wektor \\ \begin{spacing}{0.3}  \end{spacing}\\



\\~\\ 
\begin{spacing}{1.1}
\noindent{\Large Lista COP1} \\ \begin{spacing}{0.5}  \end{spacing}

\textbf{Nazwa}\\  \emph{cop1ViolationList} \\ \begin{spacing}{0.3}  \end{spacing}
 
\textbf{Opis}\\ Wyznacza listę indeksów macierzy, które nie spełniają pierwszego warunku \textit{Condition of Order Preservation (COP1)} \\  \begin{spacing}{0.3}  \end{spacing}
 
\textbf{Argumenty} \\
matrix -- macierz PC \\
resultList -- wektor wag macierzy \\  \begin{spacing}{0.3}  \end{spacing}

\textbf{Zwracana wartość}\\ Lista indeksów niespełniających $COP1$ \\ \begin{spacing}{0.3}  \end{spacing}\\


\newpage
\begin{spacing}{1.2}
\noindent{\Large COP1} \\ \begin{spacing}{0.5}  \end{spacing}

\textbf{Nazwa}\\  \emph{cop1Check} \\ \begin{spacing}{0.3}  \end{spacing}
 
\textbf{Opis}\\ Sprawdza czy każda para indeksów macierzy spełnia pierwszy warunek \textit{Condition of Order Preservation (COP1)} \\  \begin{spacing}{0.3}  \end{spacing}
 
\textbf{Argumenty} \\
matrix -- macierz PC \\  
resultList -- wektor wag macierzy \\  \begin{spacing}{0.3}  \end{spacing}

\textbf{Zwracana wartość}\\ \textit{true} jeśli $COP1$ jest spełniony, w przeciwnym razie \textit{false}\\ \begin{spacing}{0.3}  \end{spacing}\\
 

\\~\\
\begin{spacing}{1.1}
\noindent{\Large Lista COP2} \\ \begin{spacing}{0.5}  \end{spacing}

\textbf{Nazwa}\\  \emph{cop2ViolationList} \\ \begin{spacing}{0.3}  \end{spacing}
 
\textbf{Opis}\\ Wyznacza listę indeksów macierzy, które nie spełniają drugiego warunku \textit{Condition of Order Preservation (COP2)} \\  \begin{spacing}{0.3}  \end{spacing}
 
\textbf{Argumenty} \\
matrix -- macierz PC \\ 
resultList -- wektor wag macierzy \\  \begin{spacing}{0.3}  \end{spacing}

\textbf{Zwracana wartość}\\ Lista indeksów niespełniających $COP2$ \\ \begin{spacing}{0.3}  \end{spacing}\\



\\~\\ 
\begin{spacing}{1.2}
\noindent{\Large COP2} \\ \begin{spacing}{0.5}  \end{spacing}

\textbf{Nazwa}\\  \emph{cop2Check} \\ \begin{spacing}{0.3}  \end{spacing}
 
\textbf{Opis}\\ Sprawdza czy każda para indeksów macierzy spełnia drugi warunek \textit{Condition of Order Preservation (COP2)} \\  \begin{spacing}{0.3}  \end{spacing}
 
\textbf{Argumenty} \\
matrix -- macierz PC \\  
resultList -- wektor wag macierzy \\  \begin{spacing}{0.3}  \end{spacing}

\textbf{Zwracana wartość}\\ \textit{true} jeśli $COP2$ jest spełniony, w przeciwnym razie \textit{false}\\ \begin{spacing}{0.3}  \end{spacing}\\



\newpage
\begin{spacing}{1.2}
\noindent{\Large Rozbieżność rankingu} \\ \begin{spacing}{0.5}  \end{spacing}

\textbf{Nazwa}\\  \emph{errorMatrix} \\ \begin{spacing}{0.3}  \end{spacing}
 
\textbf{Opis}\\ Oblicza rozbieżność dla każdego elementu macierzy i tworzy macierz $E$ zawierającą elementy obliczane według wzoru $e_{ij} = m_{ij}\frac{r_i}{r_j}$. Jeśli macierz M nie zawiera elementów rozbieżnych (warunek \textit{COP2} jest spełniony), każdy element $e_{ij}$ wynosi $1$. \\  \begin{spacing}{0.3}  \end{spacing}
 
\textbf{Argumenty} \\
matrix -- macierz PC \\ 
resultList -- wektor wag macierzy \\  \begin{spacing}{0.3}  \end{spacing}

\textbf{Zwracana wartość}\\ Macierz błędów $E=[e_{ij}]$ \end{spacing}\\

\\~\\
\begin{spacing}{1.2}
\noindent{\Large Lokalna rozbieżność rankingu} \\ \begin{spacing}{0.5}  \end{spacing}

\textbf{Nazwa}\\  \emph{localDiscrepancyMatrix} \\ \begin{spacing}{0.3}  \end{spacing}
 
\textbf{Opis}\\ Oblicza wielkość rozbieżności (patrz \textit{errorMatrix}) dla każdego elementu. \\  \begin{spacing}{0.3}  \end{spacing}
 
\textbf{Argumenty} \\
matrix -- macierz PC \\
resultList -- wektor wag macierzy \\  \begin{spacing}{0.3}  \end{spacing}

\textbf{Zwracana wartość}\\ Macierz z lokalnymi rozbieżnościami \end{spacing}\\


\\~\\ 
\begin{spacing}{1.2}
\noindent{\Large Globalna rozbieżność rankingu} \\ \begin{spacing}{0.5}  \end{spacing}

\textbf{Nazwa}\\  \emph{globalDiscrepancy} \\ \begin{spacing}{0.3}  \end{spacing}
 
\textbf{Opis}\\ Znajduje największą lokalną rozbieżność (patrz \textit{localDiscrepancyMatrix}). \\  \begin{spacing}{0.3}  \end{spacing}
 
\textbf{Argumenty} \\
matrix -- macierz PC \\ 
resultList -- wektor wag macierzy \\  \begin{spacing}{0.3}  \end{spacing}

\textbf{Zwracana wartość}\\ Maksymalna wartość lokalnej rozbieżności \end{spacing}\\



\newpage
\begin{spacing}{1.2}
\noindent{\Large Odległość między wektorami} \\ \begin{spacing}{0.5}  \end{spacing}

\textbf{Nazwa}\\  \emph{kendallTauDistance} \\ \begin{spacing}{0.3}  \end{spacing}
 
\textbf{Opis}\\ Oblicza odległość Kendall Tau (sortowanie bąbelkowe) pomiędzy dwoma wektorami. \\  \begin{spacing}{0.3}  \end{spacing}
 
\textbf{Argumenty} \\
list1 - pierwszy wektor do porównania\\  
list2 - drugi wektor do porównania \\ \begin{spacing}{0.3}  \end{spacing}

\textbf{Zwracana wartość}\\ Liczba \textit{zamian miejscami}, które należy wkonać, aby kolejność elementów w wektorach była identyczna \\ \begin{spacing}{0.3}  \end{spacing}\\


\\~\\ 
\begin{spacing}{1.2}
\noindent{\Large Znormalizowana odległość między wektorami} \\ \begin{spacing}{0.5}  \end{spacing}

\textbf{Nazwa}\\  \emph{normalizedKendallTauDistance} \\ \begin{spacing}{0.3}  \end{spacing}
 
\textbf{Opis}\\ Oblicza odległość Kendall Tau (sortowanie bąbelkowe) pomiędzy dwoma wektorami (patrz \textit{kendallTauDistance} i dzieli je przez liczbę wszystkich możliwych \textit{zamian miejscami}. \\  \begin{spacing}{0.3}  \end{spacing}
 
\textbf{Argumenty} \\
list1 - pierwszy wektor do porównania\\  
list2 - drugi wektor do porównania \\ \begin{spacing}{0.3}  \end{spacing}

\textbf{Zwracana wartość}\\ Stosunek liczba \textit{zamian miejscami}, które należy wkonać, aby kolejność elementów w wektorach była identyczna do liczby wszystkich możliwych \textit{zamian miejscami} \\ \begin{spacing}{0.3}  \end{spacing}\\




\\~\\ 
\begin{spacing}{1.2}
\noindent{\Large Usuń wiersze } \\ \begin{spacing}{0.5}  \end{spacing}

\textbf{Nazwa}\\  \emph{deleteRows} \\ \begin{spacing}{0.3}  \end{spacing}
 
\textbf{Opis}\\ Usuwa wybrane wiersze z macierzy \\  \begin{spacing}{0.3}  \end{spacing}
 
\textbf{Argumenty} \\
matrix -- macierz PC \\ 
listOfRows -- wektor indeksów wierszy, które należy usunąć \\ \begin{spacing}{0.3}  \end{spacing}

\textbf{Zwracana wartość}\\ Macierz po usunięciu wskazanych wierszy \\ \begin{spacing}{0.3}  \end{spacing}\\


\newpage
\begin{spacing}{1.2}
\noindent{\Large Usuń kolumny } \\ \begin{spacing}{0.5}  \end{spacing}

\textbf{Nazwa}\\  \emph{deleteColumns} \\ \begin{spacing}{0.3}  \end{spacing}
 
\textbf{Opis}\\ Usuwa wybrane kolumny z macierzy \\  \begin{spacing}{0.3}  \end{spacing}
 
\textbf{Argumenty} \\
matrix -- macierz PC \\ 
listOfColumns -- wektor indeksów kolumn, które należy usunąć \\ \begin{spacing}{0.3}  \end{spacing}

\textbf{Zwracana wartość}\\ Macierz po usunięciu wskazanych kolumn \\ \begin{spacing}{0.3}  \end{spacing}\\

\\~\\
\begin{spacing}{1.2}
\noindent{\Large Usuń wiersze i kolumny } \\ \begin{spacing}{0.5}  \end{spacing}

\textbf{Nazwa}\\  \emph{deleteRowsAndColumns} \\ \begin{spacing}{0.3}  \end{spacing}
 
\textbf{Opis}\\ Usuwa wybrane wiersze i kolumny z macierzy \\  \begin{spacing}{0.3}  \end{spacing}
 
\textbf{Argumenty} \\
matrix -- macierz PC \\ 
listOfRowsAndColumns -- wektor indeksów wierszy i kolumn, które należy usunąć \\ \begin{spacing}{0.3}  \end{spacing}

\textbf{Zwracana wartość}\\ Macierz po usunięciu wskazanych wierszy i kolumn \\ \begin{spacing}{0.3}  \end{spacing}\\


\\~\\ 
\begin{spacing}{1.2}
\noindent{\Large Zwróć element macierzy } \\ \begin{spacing}{0.5}  \end{spacing}

\textbf{Nazwa}\\  \emph{getMatrixEntry} \\ \begin{spacing}{0.3}  \end{spacing}
 
\textbf{Opis}\\ Zwraca $[r,c]$ element z macierzy.\\  \begin{spacing}{0.3}  \end{spacing}
 
\textbf{Argumenty} \\
matrix -- macierz PC \\ 
r -- numer rzędu \\
c -- numer kolumny \\ \begin{spacing}{0.3}  \end{spacing}

\textbf{Zwracana wartość}\\ Wskazany element macierzy \\ \begin{spacing}{0.3}  \end{spacing}\\


\newpage
\begin{spacing}{1.2}
\noindent{\Large Utwórz kompletną macierz PC} \\ \begin{spacing}{0.5}  \end{spacing}

\textbf{Nazwa}\\  \emph{recreatePCMatrix} \\ \begin{spacing}{0.3}  \end{spacing}
 
\textbf{Opis}\\ Na podstawie macierzy z uzupełnionymi wartościami nad przekątną, tworzy kompletną macierz PC \\  \begin{spacing}{0.3}  \end{spacing}
 
\textbf{Argumenty} \\
matrix -- macierz PC \\  \begin{spacing}{0.3}  \end{spacing}

\textbf{Zwracana wartość}\\ Kompletna macierz PC \\ \begin{spacing}{0.3}  \end{spacing}\\




\\~\\ 
\begin{spacing}{1.3}
\noindent{\Large Puste elementy macierzy} \\ \begin{spacing}{0.5}  \end{spacing}

\textbf{Nazwa}\\  \emph{harkerMatrixPlaceHolderCount} \\ \begin{spacing}{0.3}  \end{spacing}
 
\textbf{Opis}\\ Sprawdza ile elementów w rzędzie ma wartość $0$. \\  \begin{spacing}{0.3}  \end{spacing}
 
\textbf{Argumenty} \\
matrix -- macierz PC \\ 
row -- numer rzędu, którego elementy są sprawdzane \\  \begin{spacing}{0.3}  \end{spacing}

\textbf{Zwracana wartość}\\ Ilość pustych elementów w wierszu \\ \begin{spacing}{0.3}  \end{spacing}\\


\\~\\ 
\begin{spacing}{1.3}
\noindent{\Large Napraw macierz} \\ \begin{spacing}{0.5}  \end{spacing}

\textbf{Nazwa}\\  \emph{harkerMatrix} \\ \begin{spacing}{0.3}  \end{spacing}
 
\textbf{Opis}\\ Tworzy macierz gotową do użycia w metodach porównań parami. Niewłaściwe elementy zastępowane są wartością $0$, a na przekątnej ustawiana jest wartość $1$. \\  \begin{spacing}{0.3}  \end{spacing}
 
\textbf{Argumenty} \\
matrix -- macierz z możliwymi błędnymi wartościami \\ \begin{spacing}{0.3}  \end{spacing}

\textbf{Zwracana wartość}\\ Macierz PC gotowa do użycia w metodach porównań parami. \\ \begin{spacing}{0.3}  \end{spacing}\\



\newpage
\begin{spacing}{1.3}
\noindent{\Large Utwórz spójną macierz} \\ \begin{spacing}{0.5}  \end{spacing}

\textbf{Nazwa}\\  \emph{consistentMatrixFromRank} \\ \begin{spacing}{0.3}  \end{spacing}
 
\textbf{Opis}\\ Na podstawie rankingu wag $W$ tworzy spójną macierz PC, której elemnty to $m_{ij}=\frac{w_i}{w_j}$. \\  \begin{spacing}{0.3}  \end{spacing}
 
\textbf{Argumenty} \\
rankList -- wektor wag macierzy \\  \begin{spacing}{0.3}  \end{spacing}

\textbf{Zwracana wartość}\\ Spójna macierz PC \\ \begin{spacing}{0.3}  \end{spacing}\\

\\~\\
\begin{spacing}{1.2}
\noindent{\Large Sortuj ranking } \\ \begin{spacing}{0.5}  \end{spacing}

\textbf{Nazwa}\\  \emph{rankOrder} \\ \begin{spacing}{0.3}  \end{spacing}
 
\textbf{Opis}\\ Sortuje malejąco wartości wektora wag od najwyższej do najniższej\\  \begin{spacing}{0.3}  \end{spacing}
 
\textbf{Argumenty} \\
rankList -- wektor wag macierzy \\  \begin{spacing}{0.3}  \end{spacing}

\textbf{Zwracana wartość}\\ Posortowany wektor wag macierzy \\ \begin{spacing}{0.3}  \end{spacing}\\
