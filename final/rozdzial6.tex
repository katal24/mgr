\chapter{Summary}
\label{sec:summary}

The goal of the work - testing common inconsistency indexes for incomplete PC matrices was achieved. One has examined sixteen different methods for calculating inconsistency. The performed teststook into account many factors which could have impact on results. It considered matrices with different size and the level of inconsistency. They were checked for varying degree of incompleteness (from $4\%$ to $50\%$). All results was gathered and presented in tables.

All tests was developed in \textit{R} language which is appropriate for numerical calculations and it fulfilled the task. Functions, which are responsible for calculations, was described and documented in details. They can compute inconsistency indexes for both full and incomplete matrices. If applicable, one can implement another indexes easily.

The tests have obtained expressly that some of existing indexes are up to calculating inconsistency for incomplete matrices after slight modifications. The methods of these modifications are presented in this work. Obviously there are many different possibilities to make changes in these indexes. Perhaps they can give even more favorable results. However, the purpose of this work was testing only existing indexes, therefore, one has decided not to make many modifications. The most reliable results was achieved by \textit{Koczkodaj index}, \textit{KulakowskiSzybowskiIa}, \textit{KulakowskiSzybowskiIab} and \textit{Cavallo DAapuzzo}. Selection one from these indexes should be done based on matrix parameters which was described in this work and presented in the results.

Very interesting came out indexes proposed by \textit{Kułakowski} and \textit{Szybowski} containing parameters $\alpha$ and $\beta$. Only one assignment of these arguments was checked in the performed tests and it gave promising results. It seems that further experiments of these indexes should be dedicated to choice of values of parameters $\alpha$ and $\beta$ in a way that will give even better results. How one was mentioned in this work, these parameters allow to choice between a more reliable result and sure that the relative error will be known.

An interesting line of enquiry can be using PC method for incomplete matrices when somehow a part of comparisons is missing. Such works already appear. However, this paper shows that also calculating inconsistency for such matrices is possible. The PC method expands for many years, still discoverS new potential and tests new places where IT can be applied.