\chapter{Summary}
\label{sec:summary}

The goal of the work - testing common inconsistency indexes for incomplete PC matrices was achieved. Sixteen different methods for calculating inconsistency were examined. The performed tests took into account many factors which could have an impact on results. It considered matrices with different sizes and the levels of inconsistency. They were checked for various degree of incompleteness (from $4\%$ to $50\%$). All results are gathered and presented in the tables.

All tests were implemented in \textit{R} language which is appropriate for numerical calculations and it fulfilled the task. Functions, which are responsible for calculations, are described and documented in details. They can compute the inconsistency indexes for both full and incomplete matrices. If applicable, one can implement another indexes easily.

The tests obtained expressly that some of existing indexes are up to calculating inconsistency for incomplete matrices after slight modifications. The methods of these modifications are presented in this work. Obviously there are many different possibilities to make changes in these indexes. Perhaps they can give even more favorable results. However, the purpose of this work was testing only existing indexes, therefore, it was decided not to make many modifications. The most reliable results were achieved by $K$, $I_{\alpha}$, $I_{\alpha,\beta}$ and $I_{CD}$. Selection one of these indexes should be done basing on matrix parameters which are described in this work and presented in the results.

Indexes proposed by \textit{Kułakowski} and \textit{Szybowski} ($I_{\alpha}$,$I_{\alpha,\beta}$) containing parameters $\alpha$ and $\beta$ came out to be very interesting. Only one assignment of these arguments was checked in the performed tests and it gave promising results. It seems that further experiments of these indexes should be dedicated to choice of values of parameters $\alpha$ and $\beta$ in a way that will give even better results. How it was mentioned in this work, these indexes could be a safer alternative to the index $K$.

An interesting line of enquiry can use PC method for incomplete matrices when somehow a part of comparisons is missing. Such works have already appeared. However, this paper shows that also calculating inconsistency for such matrices is possible. The PC method has been expanding for many years, still discovering new potential and testing new places where it can be applied.