\chapter{Summary}
\label{sec:summary}

The aim of the work - testing common inconsistency indexes for incomplete PC matrixes has been achieved. One has examined sixteen different methods for calculating inconsistency. The performed tests have consulted many factors which could have impact on results. It considered matrices with different size and the level of inconsistency. They was checked for varying degree of incompleteness (from $4\%$ to $50\%$). All results has been gathered and presented in tables.

All tests has been developed in \textit{R} language which is appropriate for numerical calculations and it has fulfilled the task. Functions, which are responsible for calculations, has been described and documented in details. They can compute inconsistency indexes for both full and incomplete matrices. If applicable, one can implement another indexes easily.

The tests has obtained expressly that some of existing indexes are up to calculating inconsistency for incomplete matrices after slight modifications. The ways of this modifications are presented in this work. Obviously there are many different possibilities to make changes in these indexes. Perhaps they can give even more favorable results. However, the purpose og this work was testing only existing indexes, therefore, one has decided to do not make many modifications. The most reliable results has been achieved by \textit{Koczkodaj index}, \textit{KulakowskiSzybowskiIa}, \textit{KulakowskiSzybowskiIab} and \textit{Cavallo DAapuzzo}. Selection one from these indexes should be done based on matrix parameters which has been described in this work and which has been presented in results.

Very interesting came out indexes proposed by \textit{Kułakowski} and \textit{Szybowski} containing parameters $\alpha$ and $\beta$. Only one assignment these arguments was checked in the performed tests. It seems that further experiments of these indexes should be dedicated to choice values of parameters $\alpha$ and $\beta$ in a way that will give even better results. How one has been mentioned in this work, these parameters allow to choice between a more reliable result and sure that the relative error will be known.

The interesting line of enquiry can be using PC method for incomplete matrixes when somehow a part of comparisons is missing. Such a works already appear. However, this paper shows that also counting inconsistency for such matrices is possible. The PC method expands for many years, still discover new potential and tests new places where can be applied.