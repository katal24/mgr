\chapter{Studies of inconsistency indexes for incomplete matrices}
\label{sec:studiesOfInconsistencyIndexesForIncompleteMatrices}

The presented inconsistency indexes have been tested. Their aim was to select those indexes which will give reliable results for incomplete matrices. Therefore, it was decided that the measure of the indexes' quality would be a \textit{relative error} (expressed as a percentage), which took into account the value of the index for a full, inconsistent matrix and the value of the index for the same matrix after partial decomposition. To be sure that the results were fair, all indexes were tested on the same set of matrices. The different sizes of the matrices, the levels of incompleteness and the levels of inconsistency were taken into account. Then, in order to compare the indexes easily and to select the best ones, the results were averaged using the arithmetic mean. While building the algorithm to solve the problem (Kazibudzki, 2017) was used.

\section{Algorithm}
\textbf{The algorithm of test of inconsistency indexes:}
\begin{enumerate}
\item Randomly generate a vector $w=[w_{1},...,w_{n}]$ and a consistent \textit{PCM matrix} associated with it $PCM=\left(m_{ij}\right)$, where $m_{ij}=\frac{w_{i}}{w_{j}}$.
\item Disrupt the matrix by multiplying its elements (excluding the diagonal) by the value of $d$, randomly selected from the range $\left(\frac{1}{x},x\right)$.
\item Replace values $m_{ij}$, where $i<j$ by values $m_{ji}$.

\item Calculate values of index with all methods for the created matrix.

\item Remove some values from the matrix by removing some of values. The level of incompleteness should be $g$\%.

\item Calculate the values of inconsistencies by all methods for the decomposed matrix.

\item Calculate the relative error for each index.

\item Repeat steps 1 to 10 $X_{1}$ times.

\item Calculate the average relative error for each inconsistency index for the \textit{PCM matrix}.

\item Repeat steps 1 to 10 $X_{2}$ times.

\item Calculate the average relative error for each index by averaging the values obtained in step 9.

\end{enumerate}


\section{Details of algorithm}
The above algorithm was carried out for values $X_{1}=100$, $X_{2}=100$. Tests were started for values d in the range $\left(1.1,1.2,...,4\right)$ and then the results were averaged. It means that the average relative error of one index was calculated on the basis of 4000 matrices, each of which decomposed randomly 100 times. It gave together 400000 tests how good the index was. 
\\

In addition, tests were carried out for various sizes of matrices.\\
\textbf{The results are divided into two parts:}
\begin{enumerate}
  \item A constant degree of incompleteness, different size of the matrix.
  \item Different degrees of incompleteness, constant size of the matrix.
\end{enumerate}

The aim of such a division is to pay attention to how the inconsistency indexes behave when the size of the matrix and the degree of incompleteness are changing. The results of the research are presented below.


\section{Implementation}
