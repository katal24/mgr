
\chapter{Heuristic Rating Estimation}
\label{sec:HRE}

\section{Wstęp do HRE}
\label{subsec:wstepHRE}
Istnieją również inne decyzje i problemy do rozwiązania. To sytuacje, w których pewne wartości są niepodważalne, z góry określone lub narzucone przez kogoś. Nie chcemy nimi manipulować ani dyskutować z ich wiarygodnością. Zależy nam natomiast, aby do tej relacji wprowadzić nowe obiekty, których wartość (w szerokim tego słowa znaczeniu) określimy w tej samej skali. Jako przykład możemy wyobrazić sobie targ z owocami, na którym sadownicy wymieniają się zbiorami bez użycia pieniędzy. Jedyną ustaloną i tradycyjną już wymianą jest sprzedawanie dwóch gruszek w zamian za trzy jabłka. Trzymając się tego wyznacznika, chcemy określić \textit{ceny} innych owoców. Jak tego dokonać? 

W takiej sytuacji z pomocą przychodzi nam wprowadzone przez doktora Kułakowskiego \textit{Heuristic Rating Estimation (HRE)}.
W mojej pracy ogólnie omówię to zagadnienie, zainteresowanych szczegółami odsyłam do źródła~\cite{A2}~\cite{A3}.

\section{Zbiór alternatyw}
\label{subsec:zbiorAlternatyw}
Pierwszym ważnym elementem HRE jest zbiór alternatyw. W tej metodzie nie będzie on już tylko zbiorem cech lub obiektów, których wartości poszukujemy. HRE zakłada, że pewne wartości są z góry określone. Nie chcemy ich zmieniać, w szczególności manipulować ich wzajemną relacją. Waga tych obiektów pozostanie niezmienna w czasie obliczeń. Drugą część zbioru tworzą jednak alternatywy, których wartości poszukujemy. Interesuje nas ich stosunek względem siebie i względem wag, które znamy.\\ 
    Jeśli więc poprzez $C_{K}$ oznaczymy alternatywy, których wartości są znane od początku (\textit{known concepts}), a poprzez $C_{U}$ opcje, których wartości chcemy przybliżyć (\textit{unknown concepts}), to zbiór alternatyw określamy jako sumę wszystkich elementów i zapisujemy jako:
$$C = C_{K} \cup C_{U}$$

\section{Macierz PC w metodzie HRE}
\label{subsec:macierzHRE}
Drugim krokiem, podobnie jak w AHP, jest stworzenie macierzy porównań parowych, która posłuży do dalszych obliczeń. Przykładowa macierz PC wygląda następująco:
$$
M = 
\left(
\begin{array}{1111}
	1 & m_{12} & m_{13} & m_{14}\\
	m_{21} & 1 & m_{22} & m_{24}\\
	m_{31} & m_{32} & 1 & \frac{2}{3}\\
	m_{41} & m_{42} & \frac{3}{2} & 1 	
\end{array}
\right)
$$
Przedstawiona macierz obrazuje sytuację, w której znamy wartości trzeciej i czwartej alternatywy, łatwo więc obliczamy ich stosunek, który wpisujemy bezpośrednio do macierzy. Z tak przygotowanej macierzy generujemy porównania parowe, które należy wykonać, a wyniki wpisujemy w odpowiednie miejsca. Porównań, w zależności od sytuacji, możemy dokonać sami lub poprosić o nie ekspertów w danej dziedzinie. Oczywiście pomijamy te, które są już znane, a więc w naszym przypadku nie będziemy szacować stosunku alternatywy trzeciej do czwartej.
	
\section{Metody HRE}
\label{subsec:metodyHRE} 
Kolejnym i zarazem najważniejszym krokiem omawianego problemu są obliczenia, które należy wykonać na uzupełnionej macierzy PC. Z racji na ich stopień zaawansowania oraz fakt, że bardzo czytelnie i~szczegółowo zostały przedstawione w źródle, nie będę w pełni ich przytaczał. Pokażę tylko kilka etapów składających się na tę metodę i do których odpowiednie funkcje zostały zaimplementowane w bibliotece.

Głównym zagadnieniem jest sprowadzenie problemu to równania postaci
$$Aw = b,$$
gdzie:\\
$A$ to macierz $r \times r$, gdzie $r$ to ilość elementów poszukiwanych, oznaczamy $|C_{U}|$, \\
$b$ to wektor wartości wyliczonych na podstawie znanych alternatyw,\\
$w$ to poszukiwany wektor wag postaci
$$w =
\left[
\begin{array}{111}
	w(c_{1}) \\ \vdots \\ w(c_{U})
\end{array}
\right] $$

Po wyliczeniu przedstawionego równania, wektor $w$ uzupełniamy o wartości ze zbioru $C_K$ i~otrzymujemy ostateczny rezultat. Należy pamiętać, że suma elementów w wyliczonym wektorze $w$ jest różna od $1$, gdyż wartości znane nie zmieniły się, alternatywy wyliczone zaś są również podane w odniesieniu do nich. Jeśli zależy nam, aby suma elementów wyniosła $1$, możemy przeskalować wektor, dzieląc każdą wartość przez sumę wszystkich elementów.
\\~\\
\begin{example}
Prowadzimy handel wymienny owoców. Wiemy, że gruszka jest $1.5$ razy cenniejsza od jabłka. W zbiorach mamy jeszcze brzoskwienie, truskawki i maliny. Chcemy oszacować wartości wszystkich owoców. 
Sporządzamy więc zbiory alternatyw znanych, nieznanych i wszystkich razem.
\begin{center} $C_{K} = [jabłko, gruszka]$, \quad \quad \quad \quad $C_{U} = [malina, brzoskwinia, truskawka]$. \\
$C = [jabłko, gruszka, malina, brzoskwinia, truskawka]$ \end{center}
\\*
Budujemy macierz PC. Przeprowadzamy odpowiednie porównania parami, a wyniki zapisujemy do macierzy.:
$$M = 
\left(
\begin{array}{11111}
	1 & \frac{2}{3} & 10 & 4 & 7\\
	\frac{3}{2} & 1 & 15 & 3 & 5\\
	\frac{1}{10} & \frac{1}{15} & 1 & \frac{1}{3} & \frac{1}{2}
	\frac{1}{4} & \frac{1}{3} & 3 & 1 & 2\\ 	
	\frac{1}{7} & \frac{1}{5} & 2 & \frac{1}{2} & 1
\end{array}
\right)
$$
Korzystając z funkcji HRE możemy od razu wyliczyć wektor wag, aby zaprezentować jednak sposób działania metody, prześledźmy kolejne etapy powstawania wyniku.\\
Wykorzystujemy algorytmy do obliczenia macierzy $A$, wektora $b$, a następnie $w$:
$$ Ab = w $$
$$    A = \left(
\begin{array}{111}
	1 & -\frac{1}{12} & -\frac{1}{8}\\
	-\frac{3}{4} & 1 & -\frac{1}{2}\\
	-\frac{1}{2} & -\frac{1}{8} & 1\\ 	
\end{array}
\right)  \quad \quad \quad \quad \quad  b = \left[
\begin{array}{1}
	0.1	\\	0.375 \\ 0.2214286
\end{array}
\right] $$
\\*
Po rozwiązaniu otrzymujemy: $$w = \left[
\begin{array}{1}
	0.22	\\	0.75 \\ 0.42
\end{array}
\right].$$
\\* 
Dodajemy wartości znane i otrzymujemy ostateczny rezultat:
$$\hat w = \left[
\begin{array}{1}
	2 \\ 3 \\ 0.2150966	\\	0.7475316 \\ 0.4224183
\end{array}
\right].$$
\end{example}
\\~\\

Ostatnią rzeczą, na którą należy zwrócić uwagę jest kwestia wyboru metod, które służą do obliczeń. W Heuristic Rating Estimation, podobnie jak w AHP możemy wybrać sposób liczenia oparty na wartościach własnych macierzy lub średnich geometrycznych. Obie drogi dają bardzo zbliżone rezultaty. W~przykładzie posłużyłem się metodą z wykorzystaniem wartości własnych.


\section{Biblioteka PairwiseComparisons - HRE}
\label{subsec:fun3}

\textbf{Funkcje z biblioteki PairwiseComparisons pomocne w powyższych obliczeniach:}
\\ 
\begin{spacing}{1.0}
\noindent{\Large Macierz $A$ metody HRE} \\ \begin{spacing}{0.5}  \end{spacing}

\textbf{Nazwa}\\ \emph{HREmatrix} \\ \begin{spacing}{0.3}  \end{spacing}
 
\textbf{Opis}\\ W oparciu o wartości własne macierzy, oblicza macierz, która razem z wektorem $b$ (patrz \textit{HREconstantTermVector}) utworzy układ równań liniowych postaci $Aw = b$.\\  \begin{spacing}{0.3}  \end{spacing}
 
\textbf{Argumenty} \\
matrix -- PC matrix \\
knownVector -- wektor znanych alternatyw, pozostałe oznaczone jako $0$ \\ \begin{spacing}{0.3}  \end{spacing}

\textbf{Zwracana wartość}\\ Macierz A, służąca do utworzenia równania $Aw = b$. \\ \begin{spacing}{0.3}  \end{spacing}

\begin{spacing}{1.2}
\noindent{\Large Wektor $b$ znanych wartości metody HRE (wykorzystuje wartości własne)} \\ \begin{spacing}{0.5}  \end{spacing}

\textbf{Nazwa}\\  \emph{HREconstantTermVector} \\ \begin{spacing}{0.3}  \end{spacing}
 
\textbf{Opis}\\ Na podstawie znanych alternatyw, w oparcu o wartości własne macierzy, oblicza wektor $b$, który razem z macierzą $A$
(patrz \textit{HREmatrix}) utworzy układ równań liniowych postaci $Aw = b$. \\ \begin{spacing}{0.3}  \end{spacing}
 
\textbf{Argumenty} \\
matrix -- PC matrix \\
knownVector -- wektor znanych alternatyw, pozostałe oznaczone jako $0$ \\ \begin{spacing}{0.3}  \end{spacing}

\textbf{Zwracana wartość}\\ Wektor $b$, służący do utworzenia równania $Aw = b$. \\ \begin{spacing}{0.3}  \end{spacing}


\\~\\ 
\begin{spacing}{1.2}
\noindent{\Large Ranking nieznanych wartości HRE (wykorzystuje wartości własne)} \\ \begin{spacing}{0.5}  \end{spacing}

\textbf{Nazwa}\\  \emph{HREpartialRank} \\ \begin{spacing}{0.3}  \end{spacing}
 
\textbf{Opis}\\ W oparcu o wartości własne macierzy oblicza nieznane alternatywy. \\  \begin{spacing}{0.3}  \end{spacing}
 
\textbf{Argumenty} \\
matrix -- PC matrix \\
knownVector -- wektor znanych alternatyw, pozostałe oznaczone jako $0$ \\ \begin{spacing}{0.3}  \end{spacing}

\textbf{Zwracana wartość}\\ Wartości nieznanych alternatyw  \\ \begin{spacing}{0.3}  \end{spacing}


\\~\\ 
\begin{spacing}{1.2}
\noindent{\Large Pełny ranking HRE (wykorzystuje wartości własne)} \\ \begin{spacing}{0.5}  \end{spacing}

\textbf{Nazwa}\\  \emph{HREfullRank} \\ \begin{spacing}{0.3}  \end{spacing}
 
\textbf{Opis}\\ W oparciu o wartości własne macierzy oblicza nieznane alternatywy i dodaje je do wektora znanych wartości. \\  \begin{spacing}{0.3}  \end{spacing}
 
\textbf{Argumenty} \\
matrix -- PC matrix \\
knownVector -- wektor znanych alternatyw, pozostałe oznaczone jako $0$ \\ \begin{spacing}{0.3}  \end{spacing}

\textbf{Zwracana wartość}\\ Wektor wag znanych i nieznaych alternatyw.\\ \begin{spacing}{0.3}  \end{spacing}\\

\newpage 
\begin{spacing}{1.1}
\noindent{\Large Przeskalowany ranking HRE (wykorzystuje wartości własne)} \\ \begin{spacing}{0.5}  \end{spacing}

\textbf{Nazwa}\\ \emph{HRErescaledRank} \\ \begin{spacing}{0.3}  \end{spacing}
 
\textbf{Opis}\\ Oblicza pełny ranking HRE (patrz \textit{HREfullRank}), a następnie skaluje wynikowy wektor wag w taki sposób, aby suma elementów wyniosiła $1$. \\  \begin{spacing}{0.3}  \end{spacing}
 
\textbf{Argumenty} \\
matrix -- PC matrix \\
knownVector -- wektor znanych alternatyw, pozostałe oznaczone jako $0$ \\ \begin{spacing}{0.3}  \end{spacing}

\textbf{Zwracana wartość}\\ Przeskalowany wektor wag znanych i nieznanych alternatyw. \\ \begin{spacing}{0.3}  \end{spacing}\\

\\~\\
\begin{spacing}{1.1}
\noindent{\Large Macierz $A$ metody HRE (wykorzystuje średnie geometryczne)} \\ \begin{spacing}{0.5}  \end{spacing}

\textbf{Nazwa}\\ \emph{HREgeomMatrix} \\ \begin{spacing}{0.3}  \end{spacing}
 
\textbf{Opis}\\ W oparciu o średnie geometryczne wierszy macierzy, oblicza macierz, która razem z wektorem $b$ (patrz \textit{HREgeomConstantTermVector}) utworzy układ równań liniowych postaci $Aw = b$.\\  \begin{spacing}{0.3}  \end{spacing}
 
\textbf{Argumenty} \\
matrix -- PC matrix \\
knownVector -- wektor znanych alternatyw, pozostałe oznaczone jako $0$ \\ \begin{spacing}{0.3}  \end{spacing}

\textbf{Zwracana wartość}\\ Macierz A, służąca do utworzenia równania $Aw = b$. \\ \begin{spacing}{0.3}  \end{spacing}


\\~\\
\begin{spacing}{1.1}
\noindent{\Large Wektor $b$ znanych wartości metody HRE (wykorzystuje średnie geometryczne)} \\ \begin{spacing}{0.5}  \end{spacing}

\textbf{Nazwa}\\  \emph{HREgeomConstantTermVector} \\ \begin{spacing}{0.3}  \end{spacing}
 
\textbf{Opis}\\ Na podstawie znanych alternatyw, w oparciu o średnie geometryczne, oblicza wektor $b$, który razem z macierzą $A$
(patrz \textit{HREgeomMatrix}) utworzy układ równań liniowych postaci $Aw = b$. \\ \begin{spacing}{0.3}  \end{spacing}
 
\textbf{Argumenty} \\
matrix -- PC matrix \\
knownVector -- wektor znanych alternatyw, pozostałe oznaczone jako $0$ \\ \begin{spacing}{0.3}  \end{spacing}

\textbf{Zwracana wartość}\\ Wektor $b$, służący do utworzenia równania $Aw = b$. \\ \begin{spacing}{0.3}  \end{spacing}


\newpage
\begin{spacing}{1.2}
\noindent{\Large Pośredni ranking nieznanych wartości HRE} \\ \begin{spacing}{0.5}  \end{spacing}

\textbf{Nazwa}\\  \emph{HREgeomIntermediateRank} \\ \begin{spacing}{0.3}  \end{spacing}
 
\textbf{Opis}\\ W oparciu o średnie geometryczne oblicza podstawę, która posłuży do obliczenia wartości nieznanych alternatyw. Współczynniki te zostaną przemnożone przez 10 (patrz \textit{HREgeomPartialRank}) \\  \begin{spacing}{0.3}  \end{spacing}
 
\textbf{Argumenty} \\
matrix -- PC matrix \\
knownVector -- wektor znanych alternatyw, pozostałe oznaczone jako $0$ \\ \begin{spacing}{0.3}  \end{spacing}

\textbf{Zwracana wartość}\\ Wektor pośrednich wartości nieznanych alternatyw  \\ \begin{spacing}{0.3}  \end{spacing}


\\~\\ 
\begin{spacing}{1.2}
\noindent{\Large Ranking nieznanych wartości HRE (wykorzystuje średnie geometryczne)} \\ \begin{spacing}{0.5}  \end{spacing}

\textbf{Nazwa}\\  \emph{HREgeomPartialRank} \\ \begin{spacing}{0.3}  \end{spacing}
 
\textbf{Opis}\\ W oparciu o średnie geometryczne oblicza nieznane alternatywy. \\  \begin{spacing}{0.3}  \end{spacing}
 
\textbf{Argumenty} \\
matrix -- PC matrix \\
knownVector -- wektor znanych alternatyw, pozostałe oznaczone jako $0$ \\ \begin{spacing}{0.3}  \end{spacing}

\textbf{Zwracana wartość}\\ Wartości nieznanych alternatyw  \\ \begin{spacing}{0.3}  \end{spacing}


\\~\\ 
\begin{spacing}{1.2}
\noindent{\Large Pełny ranking HRE (wykorzystuje średnie geometryczne)} \\ \begin{spacing}{0.5}  \end{spacing}

\textbf{Nazwa}\\  \emph{HREgeomFullRank} \\ \begin{spacing}{0.3}  \end{spacing}
 
\textbf{Opis}\\ W oparciu o średnie geometryczne oblicza nieznane alternatywy i dodaje je do wektora znanych wartości. \\  \begin{spacing}{0.3}  \end{spacing}
 
\textbf{Argumenty} \\
matrix -- PC matrix \\
knownVector -- wektor znanych alternatyw, pozostałe oznaczone jako $0$ \\ \begin{spacing}{0.3}  \end{spacing}

\textbf{Zwracana wartość}\\ Wektor wag znanych i nieznaych alternatyw.\\ \begin{spacing}{0.3}  \end{spacing}\\


\newpage
\begin{spacing}{1.2}
\noindent{\Large Przeskalowany ranking HRE (wykorzystuje średnie geometryczne)} \\ \begin{spacing}{0.5}  \end{spacing}

\textbf{Nazwa}\\ \emph{HREgeomRescaledRank} \\ \begin{spacing}{0.3}  \end{spacing}
 
\textbf{Opis}\\ Oblicza pełny ranking HRE (patrz \textit{HREgeomFullRank}), a następnie skaluje wynikowy wektor wag w taki sposób, aby suma elementów wyniosiła $1$. \\  \begin{spacing}{0.3}  \end{spacing}
 
\textbf{Argumenty} \\
matrix -- PC matrix \\
knownVector -- wektor znanych alternatyw, pozostałe oznaczone jako $0$ \\ \begin{spacing}{0.3}  \end{spacing}

\textbf{Zwracana wartość}\\ Przeskalowany wektor wag znanych i nieznanych alternatyw. \\ \begin{spacing}{0.3}  \end{spacing}\\