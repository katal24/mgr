\chapter{Introduction}
\label{cha:wprowadzenie}

\section{The Pairwise Comparisons method}
\label{sec:metodaPorowan}
People have made decisions for ages. Some of them are very simple and come easily but others, more complicated, require deeper analysis. It happens when there are many compared objects, which are complex and the selection criterion is hard to measure clearly. Fortunately, the development of mathematics brought an interesting tool - \textit{The Pairwise Comparisons (PC) Method}. The first case of using the method (in a very simple version) is the election system described by \textit{Ramond Llull} (Colomer, 2013) in the thirteenth century. Its rules were based on the fact that the candidates were pairwise compared with each other and the winner was the one who won in the largest number of direct comparisons. The method was reinvented in the eighteenth century by \textit{Condorcet} and \textit{Bord} (Kułakowski, 2016) as they proposed it in their voting system. In the twentieth century, the method found the application in the theory of social choice, the main representatives of which were the Nobel prize winners \textit{Keneth Arrow} (Arrow, 1950) and \textit{Amartya Sen} (Sen, 1977). The current shape of the method was influenced by the changes introduced by \textit{Fechner} [Fechner 1966] and then refined by {Thrustone} (Thurstone, 1994). However, the breakthrough was the introduction to the method \textit\textit{The Analytic Hierarchy Process (AHP)} by \textit{Saaty} (Saaty, 2008), which allowed to compare many more complex objects and create a hierarchical structure.The main aim of this paper is to check which method of calculating the inconsistency is the best in this case. I order to do it, a series of tests was carried out on various known inconsistency indexes, taking into account many different parameters: the matrix size, the amount of missing data, a level of inconsistency. The results of the research are included below.

The PC method is based on the assumption that it is not worth comparing all objects at the same time. It is better to compare them in pairs and then gather the results together. Such pairwise comparisons are much more intuitive and natural for a human being. How can one be sure that these judgments are consistent? Or what to do if some comparisons are missing? In such a case, is it worth taking the \textit{PC method} at all?

The answer to the first question is the concept of inconsistency introduced into the method. This paper tries to answer the next two questions - meaning to examine whether available methods for determining inconsistencies give reliable results when a part of the comparisons are missing. 

\section{Cele pracy}
\label{sec:celePracy}
The main aim of this paper is to check which method of calculating the inconsistency is the best when some comparisons are missing. I order to do it, a series of tests was carried out on various known inconsistency indexes, taking into account many different parameters: the matrix size, the amount of missing data, a level of inconsistency.
Testy zostały zaimpementowane w języku R, który dobrze nadaje się do obliczeń numerycznych.

Od początku pisania pracy nie było wiadome, jaki będzie rezultat prac. Rozważano, że któryś z istniejących współczynników niespójności okaże się odpowieni również dla macierzy niepełnych lub testy wykażą, że nie istnieje taka metoda obliczania niespójności.
The results of the research are included below.

\section{Zawartość i struktura pracy}
\label{sec:zawartoscPracy}
Praca składa się z części teoretycznej oraz oraz opisu przeprowadzonych badań i ich wyników. Łącznie daje to 6 rodziałów.\\
W drugim rodziale została zaprezentowana metoda porównywania parami oraz problem niespójności danych. Zrozumienie podstaw podstaw jest konieczne, aby przejść do kolejnych kroków.\\
W trzecim rozdziale zaprezentowano kilkanaście dostępych metod liczenia współczynnika niespójności dla macierzy porównań parowych.\\
W kolejnym rozdziale przedstawiono pomysły, dzięki którym możliwe jest przeprowadzenie testów. Są to modyfikacje istniejących współczynników niespójności, które pozwalają dostosować je do macierzy niepełnyc oraz algorytm wyznaczający jakość zmodyfikowanych współczynników. \\
W piątym rozdziale zaprezentowano wyniki badań oraz omówiono otrzymane rezultaty.\\
Ostatni rozdział to podsumowanie wykonanej pracy, wyciągnięte wnioski i pomysły na dalsze pogłębienie tematu.