\chapter{Niespójność}
\label{sec:niespojnosc}
\section{Problemy związane z metodami porównań parami}
\label{subsec:pyoblemu}

Głównym problemem metody porównań parami jest niespójność danych. To najczęstszy zarzut, jaki można usłyszeć ze strony krytyków. Być może to także jeden z powodów, dla których AHP i HRE nie zdobyły jeszcze tak dużej popularności. Mimo, iż metoda opiera się na obliczeniach matematycznych i~jest potwierdzona dowodami oraz twierdzeniami, zawiera jednak jeden \textit{słabszy} element - czynnik ludzki. Człowiek współtworzy przecież obliczenia tej metody poprzez dostarczanie wyników porównań, bez których pozostałe składniki nie mają sensu. Dlaczego tak trudno jest dostarczyć spójne dane wejściowe i czy eksperci, którzy proszeni są o dokonanie porównań nie mogliby się tego po prostu nauczyć?

Oczywiście ludzie, którzy znają zasady działania metody potrafią tak dobrać wartości macierzy PC, aby była ona spójna. Jeśli jednak widzimy tylko sparowane alternatywy i mamy przypisać im, zgodnie z naszymi odczuciami, preferencje, okazuje się to już o wiele trudniejsze. Czasem zdarzają się sytuacje, w których ciężko jest przypisać konkretne liczby do stosunku, jaki posiadamy do przedstawionych alternatyw lub określić \textit{stopień preferowania}. Skale ocen spośród których wybieramy ocenę jest umowna, a~konkretne wartości mogą zostać różnie, subiektywnie odczytane przez poszczególne osoby.
	
Zastanówmy się kiedy macierz jest niespójna. Aby łatwiej zrozumieć na czym polega problem, przedstawię dwa proste przykłady. Pierwszy z nich okaże się niespójny nawet bez zwracania uwagi na konkretne wartości.

\begin{example}Sporządzamy ranking zabawek, aby wybrać ulubiony przedmiot dziecka. Przedstawiamy mu piłkę i rower, a dziecko wybiera piłkę. Następnie pokazujemy rower i hulajnogę, wybór pada na rower. Ostatnie porównanie to hulajnoga i piłka, tym razem dziecko wskazuje na hulajnogę. \end{example}
\\*
Nie trzeba znać się na matematyce, aby szybko zorientować się, że ranking w tej sytuacji nie ma sensu, ponieważ w teorii, jeśli obiekt $A$ jest lepszy od $B$, zaś $B$ lepszy od $C$, to oczekujemy, że obiekt $A$~jest zdecydowanie bardziej preferowany niż $C$. W praktyce czasami okazuje się inaczej. W tej sytuacji dane są niespójne.

\begin{example}Porównujemy obiekt $A$ z obiektem $B$ i przypisujemy rezultat $2$. Następnie zestawiamy $B$ i $C$, tutaj również wybieramy wartość $2$. A więc, mówiąc potocznie, obiekt $A$ jest dwa razy lepszy od $B$, który z kolei $2$ razy lepszy od $C$. Jakiego rezultatu oczekujemy więc w porównaniu obiektów $A$ i $C$?\end{example}
\\*
Po chwili namysłu dochodzimy do wniosku, że obiekt $A$ w stosunku do obiektu $C$ powinien przyjąć wartość $4$. Właśnie wtedy nasze dane będą całkowicie spójne.

Drugi z przedstawionych przykładów prowadzi nas do wniosku, na którym opiera się teoria spójności danych w metodzie porównań parami:
\begin{equation}  \label{eq:1} m_{ik} = m_{ij}m_{jk} \quad \quad \quad \quad \forall_{i,j,k} \end{equation}

W praktyce okazuje się, że bardzo rzadko otrzymujemy idealnie spójną macierz, dlatego do metod porównań parami wprowadzony został współczynnik niespójności. Informuje on o stopniu niespójności danych i na jego podstawie możemy zdecydować, czy warto wykonywać obliczenia na danej macierzy PC, czy może należy poprosić o ponowne wykonanie porównań. Na przestrzeni lat powstało wiele sposobów obliczania współczynnika niespójności, w mojej pracy przedstawię dwa z nich.

\section{Współczynnik Saaty’ego}
\label{subsec:saaty}

Aby wyznaczyć współczynnik Saaty’ego należy ponownie wykorzystać maksymalną wartość własną macierzy. To właśnie w oparciu o ten parametr Saaty przedstawił swoje rozważania~\cite{A10}. Wykorzystał fakt, że największa wartość własna każdej macierzy jest równa jej wymiarowi wtedy i tylko wtedy, kiedy dana macierz jest spójna. Na tej podstawie zaproponował współczynnik niespójności (ang. \textit{Consistency Index}). Dla macierzy o wymiarze \textit{n} wyraża się wzorem:

	$$CI(A) =  \frac{ \lambda_{max} - n}{n-1} \quad $$

W najprostszej wersji obliczania współczynnika niespójności możemy w tym miejscu zakończyć nasze rozważania. Przyjmuje się, że jeżeli wyznaczona wartość CI jest mniejsza niż $0.1$, to macierz jest spójna, w~przeciwnym wypadku należy poprawić wartości porównań. 
	
\begin{example}
$$M = 
\left(
\begin{array}{1111}
	1 & 2 & 8 \\
	\frac{1}{2} & 1 & \frac{3}{4} \\
	\frac{1}{8} & \frac{4}{3} & 1 	
\end{array}
\right)$$
\\*
Wyznaczamy największą wartość własną: $\lambda_{max} = 3.319518$,
a następnie współczynnik: $$CI(M) = \frac{3.319518 - 3}{3 -1} = 0.159759. $$
\\*
Otrzymany rezultat informuje nas, że macierz $M$ nie jest spójna.
\end{example}

Nieco bardziej dokładny sposób obliczania niespójności zaproponowany przez Saaty’ego zestawia wartość $CI$ z współczynnikiem zależnym od wymiaru macierzy. Pozwala to na bardziej szczegółowe określenie wielkości niespójności danych. W tym przypadku należy wykorzystać tabelę (\ref{tab:ri})
\begin{table}[ht!]
\begin{center}


\caption{Wartości $RI_{n}$}
\label{tab:ri}
\begin{tabular}{|c|c|c|c|c|c|}
\cline{1-6} \multicolumn{1}{|l|}{$n$}
& 3 &
4 &
5 &
6 &
7 \\\hline
$RI_n$ & 0.5247 & 0.8816 & 1.1086 & 1.2479 & 1.3417\\ \hline
\end{tabular}
\end{center}
\end{table}
i wyznaczyć współczynnik nazywany \textit{Consistency Ratio} (CR) według wzoru:
$$CR(A) = \frac{CI(A)}{RI_{n}}$$
\\*
W przypadku naszej macierzy $M$ wynosi on: $\frac{0.159759}{0.5247} = 0.3044768.$ W tej metodzie również przyjmuje się, że warunkiem spójności jest spełnienie nierówności: $CR \leq 0.1$, więc klasyfikujemy macierz $M$ jako niespójną.

\section{Metoda odległościowa - Koczkodaj}
\label{subsec:koczkodaj}
Jedną z głównych wad współczynnika Saaty’ego jest fakt, że wartości własne są wielkościami charakteryzującymi całą macierz, nie pozwalają więc określić, które elementy powodują wystąpienie niespójności. Rozwiązaniem tego problemu jest wprowadzona przez Koczkodaja~\cite{A11} metoda odległościowa, którą krótko przedstawię. Nieco bardziej rozbudowany opis, napisany przyjaznym językiem, można znaleźć w~\cite{A12}.
\\*
Zacznijmy od rozważenia macierzy o wymiarach $3\times3$:
$$A = 
\left(
\begin{array}{111}
	1 & a & b \\
	\frac{1}{a} & 1 & c \\
	\frac{1}{b} & \frac{1}{c} & 1 	
\end{array}
\right)$$
Odwołując się do \ref{eq:1} możemy wnioskować, że $b = ac$.
Nasza macierz będzie spójna, jeśli spełniony zostanie ten warunek.

W celu zmierzenia ewentualnej niespójności, ponownie wykorzystując \ref{eq:1}, możemy stworzyć trzy wektory, w których jedna wartość zostanie wyliczona jako kombinacja dwóch pozostałych. Otrzymujemy więc wektory: $(\frac{b}{c}, b, c)$, $(a, ac, c)$ i $(a, b, \frac{b}{a})$. Następnie sprawdzamy odległość każdego z~tych wektorów od danego w przykładzie wektora $(a b c)$. Wybieramy ten, którego wartość odległości jest najniższa. Uzyskany rezultat to współczynnik niespójności. Zapis formalny przedstawionego algorytmu wygląda następująco:
	$$CM(a,b,c) = min \{ \frac{1}{a}|a - \frac{b}{c}|, \frac{1}{b}|b - ac|,\frac{1}{c}|c - \frac{b}{a}|\}$$

Teraz możemy przejść do macierzy o większych wymiarach. Okazuje się, że wystarczy wyszukać wszystkie trójki liczb, które powinny być od siebie zależne. Trójki te zostały nazwane \textit{triadami}. Aby wyznaczyć współczynnik niespójności macierzy wystarczy wybrać triad, dla którego wyliczona wartość jest największa:
	$$CM(A) = max\{\quadmin\{|1-\frac{b}{ac}|,|1-\frac{ac}{b}|}\} \quad \quad  \forall_{ triady \quad (a,b,c) \quad macierzy\quad A} $$
Przyjmuje się, że macierz jest spójna, jeżeli $CM(A) \leq \frac{1}{3}$.

Warto zauważyć, że metoda odległościowa pozwana nie tylko zbadać niespójność, ale także wskazuje miejsce, które ma na nią największy wpływ. Dlatego można poprawić wprowadzone do macierzy PC wartości w jednym, konkretnym miejscu i przez to zmniejszyć współczynnik niespójności.



\section{Biblioteka PairwiseComparisons - niespójność}
\label{subsec:fun4}

\textbf{Funkcje z biblioteki PairwiseComparisons pomocne w powyższych obliczeniach:}
\\
\begin{spacing}{1.1}
\noindent{\Large Współczynnik Saaty'ego} \\ \begin{spacing}{0.5}  \end{spacing}

\textbf{Nazwa}\\ \emph{saatyIdx} \\ \begin{spacing}{0.3}  \end{spacing}
 
\textbf{Opis}\\ Oblicza współczynnik niespójności macierzy w podstawowej wersji zaproponowanej przez Saaty'ego.\\  \begin{spacing}{0.3}  \end{spacing}
 
\textbf{Argumenty} \\
matrix -- PC matrix \\ \begin{spacing}{0.3}  \end{spacing}

\textbf{Zwracana wartość}\\ Wartość współczynnika Saaty'ego. \\ \begin{spacing}{0.3}  \end{spacing}

\textbf{Dodatkowe informacje} \\ W bibliotece dostępna jest również symboliczna wersja tej funkcji.\emph{saatyIdxSym}


\\~\\ 
\begin{spacing}{1.2}
\noindent{\Large Współcznnik niespójności Koczkodaja dla triady} \\ \begin{spacing}{0.5}  \end{spacing}

\textbf{Nazwa}\\  \emph{koczkodajTriadIdx} \\ \begin{spacing}{0.3}  \end{spacing}
 
\textbf{Opis}\\ Oblicza współczynnik niespójności dla triady metodą Koczkodaja\\  \begin{spacing}{0.3}  \end{spacing}
 
\textbf{Argumenty} \\
triad -- wektor trzech liczb \\ \begin{spacing}{0.3}  \end{spacing}

\textbf{Zwracana wartość}\\ Współczynnik Koczkodaja \\ \begin{spacing}{0.3}  \end{spacing}\\


\newpage
\begin{spacing}{1.2}
\noindent{\Large Najbardziej niespójny triad macierzy} \\ \begin{spacing}{0.5}  \end{spacing}

\textbf{Nazwa}\\  \emph{koczkodajTheWorstTriad} \\ \begin{spacing}{0.3}  \end{spacing}
 
\textbf{Opis}\\ Znajduje triad, którego wartość współczynnika niespójności Koczkodaja jest największa.\\  \begin{spacing}{0.3}  \end{spacing}
 
\textbf{Argumenty} \\
matrix -- PC matrix \\ \begin{spacing}{0.3}  \end{spacing}

\textbf{Zwracana wartość}\\ Triad o największym współczynniku niespójności. \\ \begin{spacing}{0.3}  \end{spacing}\\


\\~\\
\begin{spacing}{1.2}
\noindent{\Large Najbardziej niespójne triady macierzy} \\ \begin{spacing}{0.5}  \end{spacing}

\textbf{Nazwa}\\  \emph{koczkodajTheWorstTriads} \\ \begin{spacing}{0.3}  \end{spacing}
 
\textbf{Opis}\\ Znajduje triady, których wartości współczynnika niespójności Koczkodaja są największe.\\  \begin{spacing}{0.3}  \end{spacing}
 
\textbf{Argumenty} \\
matrix -- PC matrix \\ 
n -- ilość poczukiwanych triad	\\ \begin{spacing}{0.3}  \end{spacing}

\textbf{Zwracana wartość}\\ Triady o największym współczynniku niespójności. \\ \begin{spacing}{0.3}  \end{spacing}\\

\\~\\
\begin{spacing}{1.2}
\noindent{\Large Współczynnik niespójności Koczkodaja} \\ \begin{spacing}{0.5}  \end{spacing}

\textbf{Nazwa}\\ \emph{koczkodajIdx} \\ \begin{spacing}{0.3}  \end{spacing}
 
\textbf{Opis}\\ Oblicza współczynnik niespójności macierzy w podstawowej wersji zaproponowanej przez Koczkodaja.\\  \begin{spacing}{0.3}  \end{spacing}
 
\textbf{Argumenty} \\
matrix -- PC matrix \\ \begin{spacing}{0.3}  \end{spacing}

\textbf{Zwracana wartość}\\ Wartość współczynnika Koczkodaja. \\ \begin{spacing}{0.3}  \end{spacing}


\newpage
\begin{spacing}{1.2}
\noindent{\Large Spójny triad} \\ \begin{spacing}{0.5}  \end{spacing}

\textbf{Nazwa}\\  \emph{koczkodajConsistentTriad} \\ \begin{spacing}{0.3}  \end{spacing}
 
\textbf{Opis}\\ Na podstawie przekazanej trójki liczb, znajduje triad , którego wartość współczynnika niespójności jest najmniejsza.\\  \begin{spacing}{0.3}  \end{spacing}
 
\textbf{Argumenty} \\
triad -- wektor trzech liczb \\ \begin{spacing}{0.3}  \end{spacing}

\textbf{Zwracana wartość}\\ Triad o najmniejszym współczynniku niespójności. \\ \begin{spacing}{0.3}  \end{spacing}\\


\\~\\
\begin{spacing}{1.2}
\noindent{\Large Poprawiona macierz } \\ \begin{spacing}{0.5}  \end{spacing}

\textbf{Nazwa}\\  \emph{koczkodajImprovedMatrixStep} \\ \begin{spacing}{0.3}  \end{spacing}
 
\textbf{Opis}\\ Znajduje miejsce (triad), w którym macierz jest najbardziej niespójna, a następnie poprawia wartości w tych miejsach, w taki sposób, aby współczynnik niespójności zmniejszył nię \\  \begin{spacing}{0.3}  \end{spacing}
 
\textbf{Argumenty} \\
matrix -- PC matrix \\ \begin{spacing}{0.3}  \end{spacing}

\textbf{Zwracana wartość}\\ Macierz o mniejszym współczynniku niespójności. \\ \begin{spacing}{0.3}  \end{spacing}\\