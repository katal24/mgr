\chapter{Summary}
\label{sec:summary}

Cel pracy, którym było przebadanie popularnych współczynników niespójności dla niepełnych macierzy PC, został osiągnięty. Przetestowano szesnaście różnych sposobów liczenia niespójności macierzy. Przeprowadzone badania uwzględniły wiele różnych czynników, które mogły mieć wpływ na wyniki. Wzięto pod uwagę macierze o różnych wymiarach oraz różnym poziomie niekomplentości. Sprawdzono je dla różnych stopni niekomplentości (od 4\% do 50 \%). Wszystkie wyniki zebrano, a wyniki zostały zaprezentowane w tabelach.

Wszystkie testy zostały zaimplementowane w języku R, który przeznaczony jest do obliczeń numerycznych i spełnił swoje zadanie. Funkcje, które odpowiadały za obliczenia szczegółowo opisano i udokumentowano. Potrafią obliczać współczynniki niespójności zarówno dla macierzy pełnych, jak też niekompletnych. W razie potrzeby można w stosunkowo łatwy sposób zaimplementować kolejne współczynniki.

Testy jednoznacznie pokazały, że niektóre z isniejących współczynników, po lekkiej modyfikacji, nadają się do obliczania niespójności macierzy niepełnych. Sposoby modyfikacji zostały przedstawione w pracy. Oczywiście istnieje wiele innych możliwości zmian w tych współczynnikach. Być może dadzą one nawet korzystniejsze rezultaty. Celem tej pracy było jednak zbadanie istniejących współczynników, dlatego zdecydowano nie dokonowyać w nich wielu zmian. Najlepsze rezultaty osiągają \textit{Koczkodaj index}, \textit{KulakowskiSzybowskiIa}, \textit{KulakowskiSzybowskiIab}, \textit{Cavallo DAapuzzo}. Wybór jednego, konkretnego współczynnika należy podjąć w oparciu o parametry macierzy, które zostały opisane w niniejszej pracy i które prezentują wyniki.

Bardzo ciekawe okazały się współczynniki wprowadzone przez \textit{Kułakowskiego} i \textit{Szybowskiego} zawierające parametry $\alpha$ i $\beta$. W przeprowadzonych testach wykorzystano tylko jedno przypisanie tych parametrów, a współczynniki dały bardzo dobre rezultaty. Wydaje się, że kolejne badania tych współczynników warto poświęcić na dobór parametrów w taki sposób, aby wyniki były jeszcze lepsze. Jak wspomniano w pracy pozwalają one również na wybór między lepszym wynikiem, a pewnością, że rezultat będzie odpowieni.

Ciekawym kierunkiem rozwoju może być właśnie stosowanie metody dla macierzy niepełnych, kiedy z jakiegoś powodu brakuje pewnych danych. Takie prace już się ukazują. Niniejsza praca pokazuje jednak, że także liczenie niespójności dla takich macierzy jest możliwe. Metoda porównywania parami rozwija się od wielu lat, ciągle odkrywa nowe możliwości i bada kolejne miejsca, gdzie może być stosowana. Ciągle jednak istnieją obszary, w których może się rozwijać, czego dowodem jest niniejsza praca.